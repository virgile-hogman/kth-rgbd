%%
%% This is file `thesis.tex',
%% generated with the docstrip utility.
%%
%% The original source files were:
%%
%% kthesis.dtx  (with options: `ex1')
%% 
%% IMPORTANT NOTICE:
%% 
%% For the copyright see the source file.
%% 
%% Any modified versions of this file must be renamed
%% with new filenames distinct from thesis.tex.
%% 
%% For distribution of the original source see the terms
%% for copying and modification in the file kthesis.dtx.
%% 
%% This generated file may be distributed as long as the
%% original source files, as listed above, are part of the
%% same distribution. (The sources need not necessarily be
%% in the same archive or directory.)
\documentclass[a4paper,11pt]{kth-mag}
\usepackage[T1]{fontenc}
\usepackage{textcomp}
\usepackage{lmodern}
\usepackage[latin1]{inputenc}
\usepackage[swedish,english]{babel}
\usepackage{nada-ex}
\usepackage{graphicx}
\title{Building a 3D map from RGB-D sensors}
\author{Virgile H\"{o}gman}
\date{September 2011}
\blurb{Master's Thesis at CVAP\\Supervisor: Alper Aydemir\\Examiner: Stefan Carlsson}
\trita{TRITA xxx yyyy-nn}
\begin{document}
\frontmatter
\maketitle
%%
%% This is file `kth-abs.tex',
%% generated with the docstrip utility.
%%
%% The original source files were:
%%
%% kthesis.dtx  (with options: `abstract')
%% 
%% IMPORTANT NOTICE:
%% 
%% For the copyright see the source file.
%% 
%% Any modified versions of this file must be renamed
%% with new filenames distinct from kth-abs.tex.
%% 
%% For distribution of the original source see the terms
%% for copying and modification in the file kthesis.dtx.
%% 
%% This generated file may be distributed as long as the
%% original source files, as listed above, are part of the
%% same distribution. (The sources need not necessarily be
%% in the same archive or directory.)
\begin{abstract}
  This is a skeleton for KTH theses. More documentation
  regarding the KTH thesis class file can be found in
  the package documentation.
\end{abstract}
\endinput
%%
%% End of file `kth-abs.tex'.

\clearpage
\selectlanguage{swedish}
%\begin{abstract}
%  Denna fil ger ett avhandlingsskelett.
%  Mer information om \LaTeX-mallen finns i
%  dokumentationen till paketet.
%\end{abstract}
\selectlanguage{english}
\clearpage
\tableofcontents
\mainmatter
\chapter{Introduction}

Currently, most of robotic mapping is performed using sensors that offers only a 2D cross section of the environment around them. The reason for this is ways of acquiring high quality 3D data were either very expensive or had hard constraints on the robot movements. However, recently there has been a great interest in processing data acquired using depth measuring sensors due to the availability of cheap and high performance RGB-D cameras. 

\section{Context}

SLAM (Simultaneous Localization and Mapping) adresses the dual problem of building a map and locate the robot at the same time. While this process is generally based on odometry data provided by different sensors, Visual SLAM focuses on the use of camera, for instance 3D. The process is then to estimate the poses of the camera in order to reconstruct the entire environment while the camera is moving.

The main problem comes from the measurement errors that leads to deviations in the estimations of each camera poses with respect to the the real motion. Probablistic models are widely used to reduce these errors and provide satisfying estimations.

\section{Goals}

The goal of this work is to build a 3D map from RGB \& depth information provided by a camera. The hardware device used here is the Microsoft Kinect. While Visual SLAM is intended to be used in real time, some of the processing may be done without focusing on the performance issues. The main objective of this work is to have an overview of the different methods and techniques in order to build a consistent map, so the problems can better be identified and then be handled individually.

\section{Thesis outline}
The rest of the document is structured as follows:
\begin{description}
\item[Chapter 2] presents the Background and the underlying concepts.
\item[Chapter 3] presents the Method.
\item[Chapter 4] presents the Results obtained with different datasets.
\item[Chapter 5] presents the Future Works.
\end{description}

\chapter{Background}

\section{Microsoft Kinect}

The hardware used in this work is the Microsoft Kinect, a device developed by PrimeSense, initially for the Microsoft Xbox 360 and released in November 2010. It is composed by an RGB camera, 3D depth sensors, a multi-array microphone and a motorized tilt. In this work, only the RGB and depth sensors are used to provide the input data.

Kinect characteristics
The RGB sensor is a regular camera that streams video with 8 bits for every color channel, giving a 24-bit depth. Its color filter array is a Bayer filter mosaic. The compound system has VGA resolution (640x480 pixels) and a maximal frame rate of 30 Hz. The field of view is 57\textdegree horizontal 43\textdegree vertical, with a tilt range of $\pm$ 27\textdegree.

The depth sensing system is composed by an IR emitter and a standard CMOS Sensor. The IR light source projects structured light, which is then captured by the CMOS image sensor, and decoded to produce the depth image of the scene. It has an operation range between 0.7 and 6 meters, although best results are obtained in the range that goes from 1.2 to 3.5 meters. Its data output has 12-bit depth.

The drivers used are those developed by OpenNI (Open Natural Interface). OpenNI is an organization composed by PrimeSense Willow Garage, Side-Kick, ASUS.


\section{Features}

An approach widely used in computer vision, especially in object recognition, is based on the detection of points of interests on object or surfaces. This is done through the extraction of features. In order to track these points of interests during a movement of the camera and/or the robot, a reliable feature detection has to be invariant to image location, scale and rotation. A few methods are briefly presented here:
\begin{description}
\item[SIFT] Scale Invariant Feature Transform
\item[SURF] Speeded Up Robust Feature
\item[NARF] Normal Aligned Radial Feature
\end{description}

\subsection{SIFT}
David Lowe.
This is a method to detect distinctive, invariant image feature points, which easily can be matched between images to perform tasks such as object detection and recognition, or to compute geometrical transformations between images.

Several libraries exist. For this work is used the library developed by Rob Hess.

\subsection{SURF}
Source Wiki.
SURF (Speeded Up Robust Feature) is a robust image detector \& descriptor, first presented by Herbert Bay et al. in 2006, that can be used in computer vision tasks like object recognition or 3D reconstruction. It is partly inspired by the SIFT descriptor. The standard version of SURF is several times faster than SIFT and claimed by its authors to be more robust against different image transformations than SIFT. SURF is based on sums of approximated 2D Haar wavelet responses and makes an efficient use of integral images.

\subsection{NARF}
This method is presented in [ref] and is part of ROS Point Cloud Library.

\section{Matching}

\subsection{RANSAC}
RANdom SAmple Consensus. Iterative method.

Define the number of iterations.
For each iteration:
Pickup k samples randomly and determine a model (hypothesis). Evaluate this model by computing the error for all the points of the dataset. If enough points are valid (number of inliers), then the hypothesis is valid. Keep the best hypothesis with respect to the the mean error (lowest is best).

\subsection{ICP}
Iterative Closest Point. As any gradient descent method, the ICP is applicable when we have in advance an relatively good starting point. Otherwise, it will be trapped into a local minimum. One possibility, as used in Intel Labs, is to run RANSAC before, to determine the first guess for the ICP procedure.

\section{SLAM}

\subsection{EKF}

\subsection{Graph optimization}

g2o, HOG-Man, GraphSLAM

\section{Reconstruction and visualization of the map}

Point cloud. PCL.

\chapter{Method}

The approach: SIFT extraction in 2D, kd-tree matching in 2D/3D, RANSAC in 3D, g2o.

\section{Features}
\subsection{Features Extraction}

SIFT with SIFT Library.

\subsection{Features Matching}

KD-tree with SIFT Library.
Computes the initial matches between a couple of frames. Use the depth information.

\section{Transformation with RANSAC}

Done from the initial matches.
How to define the quality of transformation? 
Mean error, number of inliers, rate of inliers wrt initial matches.

\section{Graph optimization}
\subsection{Building the graph}

Compute estimations of the poses.

\subsection{Solving the graph}

Various methods proposed in g2o.

\subsection{Loop closures}

Define the criterion for loop closure.


\section{Reconstruction}

Point cloud with PCL. The positions of the cameras determine how each individual point cloud should be projected.

\chapter{Results}

Different datasets were used.

\section{CVAP (KTH)}

\subsection{1 room}
Map with 1 room and 1 loop closures.
\begin{figure}[h]
\centering
\includegraphics[width=0.8\textwidth]{figures/room1}
\caption{Map with 1 room}
\end{figure}


\subsection{2 rooms}
Map with 2 rooms and n loop closures.
\begin{figure}[h]
\centering
\includegraphics[width=0.8\textwidth]{figures/room2}
\caption{Map with 1 room}
\end{figure}

\subsection{4 rooms}

\section{other datasets}

\chapter{Future Works}

Performance improvements: SURF, SIFT GPU.

Quality improvements: ICP, graph optimization, combine sensors, define new criterions for loop closures.

Other approach:
Microsoft Fusion: no feature with real-time performance. Based on ICP, directly run on the point clouds from the depth data. Details of the method not published yet (patents?).

\chapter{Bibliography}

Intel labs
P. Henry, M. Krainin, E. Herbst, X. Ren, and D. Fox. RGB-D mapping:
Using depth cameras for dense 3D modeling of indoor environments.
In Proc. of the Intl. Symp. on Experimental Robotics (ISER), Delhi,
India, 2010.

g20
Rainer Kuemmerle, Giorgio Grisetti, Hauke Strasdat, Kurt Konolige, and Wolfram Burgard: g2o: A General Framework for Graph Optimization, IEEE International Conference on Robotics and Automation (ICRA), 2011

HOG-Man
G. Grisetti, R. K\"ummerle, C. Stachniss, U. Frese, and C. Hertzberg.
Hierarchical optimization on manifolds for online 2D and 3D mapping.
In Proc. of the IEEE Intl. Conf. on Robotics and Automation (ICRA),
Anchorage, AK, USA, 2010.

David G. Lowe, "Distinctive image features from scale-invariant keypoints," International Journal of Computer Vision, 60, 2 (2004), pp. 91-110.

SIFT Rob Hess
http://blogs.oregonstate.edu/hess/code/sift/

NARF

\end{document}
\endinput
%%
%% End of file `thesis.tex'.



